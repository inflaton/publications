\balance
\section{Conclusion and Future Work}

This study presents a novel approach for automating the collection and categorization of global maritime disruption data, utilizing the capabilities of Large Language Models (LLMs) like GPT-4o. The findings reveal that LLMs, particularly GPT-4o, significantly outperform traditional machine learning models in terms of both accuracy and consistency, proving to be more effective in classifying complex news data. The introduction of the "Ratio of Valid Categories" metric provided valuable insights into the reliability of each model, with GPT-4o achieving a perfect 100\% ratio of valid categories across all scenarios. Although other models, such as GPT-4o-mini and Meta-Llama-3.1, demonstrated some variability, the results underscore the potential of LLMs in enhancing risk data collection and research capabilities. The findings also emphasize the importance of selecting robust models and highlight the role of LLMs in streamlining data-driven risk management.

While the results of this study are promising, several limitations must be acknowledged. The primary limitation is the quality of the available data, which inherently contains ambiguities and inconsistencies due to the diverse nature of news articles. These characteristics pose significant challenges for accurate categorization.
Moreover, the use of GPT-4o for relabelling the data may introduce biases, particularly as the model is also evaluated on this dataset. To mitigate these potential biases and to ensure more robust validation, future work will incorporate comprehensive human evaluation.
% Future research will aim to address these limitations by improving data quality and refining evaluation methodologies. There is considerable scope for exploring more advanced LLM models, including fine-tuning LLMs to enhance classification accuracy further. Continued research in these areas will be crucial for advancing the field of automated maritime risk management.
Moving forward, research will focus on improving data quality and refining evaluation methodologies. Exploring more advanced LLM models and fine-tuning them for better classification accuracy will be critical for advancing automated maritime risk management.